\section{Problem 3.7}
\paragraph{(a)}
\begin{align*}
    p(X \vert \theta) 
    &\propto \theta^X (1-\theta)^{n-X}\\
    &= \exp \left\{ X\ln \theta + (n-X) \ln (1-\theta)
        \right\}\\
    &= \exp \left\{ X\ln \left( \frac{\theta}{1-\theta} \right) + n\ln(1-\theta)
        \right\}
\end{align*}
Thus, $p(X \vert \theta)$ belongs to the exponential family with the natural parameter $\zeta = \ln \left( \frac{\theta}{1-\theta}\right)$.

\paragraph{(b)}
\begin{align*}
    p(X \vert \zeta)
    &\propto \exp \left\{ \zeta X - \Phi(\zeta)
        \right\}\\
    \text{where } \Phi(\zeta)
    &= n \ln (\exp(\zeta) + 1)
\end{align*}
With $\mu, \lambda \in \mathbb{R}$ and $\lambda > 0$, the conjugate family over $\zeta$ is given by
\begin{align*}
    \mathcal{F} = \left\{ p(\zeta \vert \mu, \lambda )\right\}
    &\propto \exp \left\{ \zeta \mu - \lambda \Phi(\zeta)
        \right\}
\end{align*}

\paragraph{(c)}
With the prior $\pi(\zeta \vert \mu, \lambda)$ over $\zeta$, the corresponding posterior over $\zeta$ is
\begin{align*}
    \pi(\zeta \vert x)
    &\propto p(x \vert \zeta) \pi(\zeta)\\
    &\propto \exp \left\{ \zeta x - \Phi(\zeta) \right\} \cdot \exp \left\{ \zeta \mu - \lambda \Phi(\zeta) \right\} \\
    &= \exp \left\{ \zeta (\mu+x) - (\lambda+1) n \ln (\exp(\zeta) + 1) \right\}
\end{align*}
The equivalent posterior over $\theta$ is 
\begin{align*}
    \pi(\theta \vert x)
    &\propto \exp \left\{ \ln \left(\frac{\theta}{1-\theta}\right) (\mu+x) + (\lambda+1) n \ln (1-\theta) \right\}
\end{align*}

\paragraph{(d)}
The conjugate prior $\pi(\theta \vert \mu, \lambda)$ can be reformed as
\begin{align*}
    \pi(\theta \vert \mu, \lambda)
    &\propto  \exp \left\{ \ln \left(\frac{\theta}{1-\theta}\right) \mu + \lambda n \ln (1-\theta) \right\}\\
    &= \left( \frac{\theta}{1-\theta} \right)^\mu \cdot (1-\theta)^{\lambda n}\\
    &= \theta^\mu (1-\theta)^{\lambda n - \mu}
\end{align*}
It is recognized that the conjugate prior belongs to the Beta distribution family.