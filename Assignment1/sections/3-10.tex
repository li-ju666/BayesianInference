\section{Problem 3.10}
\paragraph{(a)}
Given $p(X \vert \theta) = \mathcal{N}(\theta, 1)$, we have
\begin{align*}
    V(\theta \vert x) 
    &= \frac{\partial p(x \vert \theta)}{p(x \vert \theta) \partial \theta}\\
    &= \frac {\partial -\frac{1}{2}(x - \theta)^2} {\partial \theta}\\
    &= x - \theta\\
    I(\theta) 
    &= -\mathbb{E}_{p(X \vert \theta)} \left[ \frac{\partial V(\theta \vert x)}{\partial \theta}\right]\\
    &= 1
\end{align*}
Thus, Jeffreys prior is given by $\pi_\mathrm{Jeff}(\theta) \propto 1$. Then the corresponding posterior is derived as follows
\begin{align*}
    \pi_\mathrm{Jeff} (\theta \vert x)
    &\propto \pi_\mathrm{Jeff} (\theta) \cdot p(x \vert \theta)\\
    &\propto \exp \left\{ -\frac{1}{2} (x - \theta)^2 \right\}
\end{align*}
It is recognized that the posterior is a Gaussian $\pi_\mathrm{Jeff} (\theta \vert x) = \mathcal{N}(x, 1)$.

\paragraph{(b)}
Given that $\pi_\mathrm{Jeff}(\theta) \propto 1$, Jeffreys prior for $\theta \in [-k, k ]$ is given by
\begin{align*}
    \pi^k_\mathrm{Jeff}(\theta) = 
    \begin{cases}
        \frac{1}{2k} & \text{for }\theta \in [-k, k]\\
        0 & \text{otherwise}
    \end{cases}
\end{align*}
The corresponding posterior is derived as follows:
\begin{align*}
    \pi_\mathrm{Jeff}^k (\theta \vert x)
    \begin{cases}
        \propto \pi_\mathrm{Jeff}^k (\theta) \cdot p(x \vert \theta)
        \propto \exp \left\{ -\frac{1}{2} (x - \theta)^2 \right\}
        &  \text{for }\theta \in [-k, k]\\
        0 & \text{otherwise}
    \end{cases}
\end{align*}
It is recognized that the posterior is a truncated Gaussian $\pi_\mathrm{Jeff}^k (\theta \vert x) = \mathcal{TN}(x, 1, -k, k)$.

\paragraph{(c)}
\begin{proof}
The Kullback–Leibler divergence between $\pi_\mathrm{Jeff} (\cdot \vert x)$ and $\pi_\mathrm{Jeff}^k (\cdot \vert x)$ over the common support $[-k, k]$ is shown as follows:
\begin{align*}
    &D_{\mathrm{KL}}\left( \pi_\mathrm{Jeff} (\cdot \vert x) \Vert \pi_\mathrm{Jeff}^k (\cdot \vert x)\right)\\
    % &= \int_{-\infty}^{\infty}  \pi_\mathrm{Jeff} (\theta \vert x)
    %     \ln \left( \frac{\pi_\mathrm{Jeff} (\theta \vert x)}{\pi_\mathrm{Jeff}^k (\theta \vert x)} \right) d\theta \\
    =& \int_{-k}^{k}  \pi_\mathrm{Jeff} (\theta \vert x)
        \ln \left( \frac{\pi_\mathrm{Jeff} (\theta \vert x)}{\pi_\mathrm{Jeff}^k (\theta \vert x)} \right) d\theta \\
        % +
        % \int\limits_{\theta \notin [-k, k]}  \pi_\mathrm{Jeff} (\theta \vert x)
        % \ln \left( \frac{\pi_\mathrm{Jeff} (\theta \vert x)}{\pi_\mathrm{Jeff}^k (\theta \vert x)} \right) d\theta \\
    =& \int_{-k}^{k}  \phi(\theta; x) \ln \left( 
        \frac{\phi(\theta; x)}{ \frac{\phi(\theta;x)}{\Phi(k-x;x) - \Phi(-k-x;x)}}
        \right)d\theta \\
        % + \int\limits_{\theta \notin [-k, k]}  \pi_\mathrm{Jeff} (\theta \vert x)
        % \ln \left( \frac{\pi_\mathrm{Jeff} (\theta \vert x)}{\pi_\mathrm{Jeff}^k (\theta \vert x)} \right) d\theta \\
    =& \ln \left( \Phi(k-x;x) - \Phi(-k-x;x) \right) \int_{-k}^{k}
        \phi(\theta;x)d\theta\\
    =& \left( \Phi(k-x;x) - \Phi(-k-x;x) \right) \ln \left( \Phi(k-x;x) - \Phi(-k-x;x) \right)\\
    \text{where}\quad
    &\phi(\theta;x) = \frac{1}{\sqrt{2\pi}}\exp \left( -\frac{(\theta - x)^2}{2}\right)\\
    &\Phi(\theta;x) = \int_{-\infty}^{\theta} \phi(z;x) dz
\end{align*}
Then we have
\begin{align*}
    &\lim_{k\rightarrow \infty} D_{\mathrm{KL}}\left( \pi_\mathrm{Jeff} (\cdot \vert x) \Vert \pi_\mathrm{Jeff}^k (\cdot \vert x)\right)\\
    =&\lim_{k\rightarrow \infty} \left( \Phi(k-x;x) - \Phi(-k-x;x) \right) \ln \left( \Phi(k-x;x) - \Phi(-k-x;x) \right)\\
    =& \lim_{k\rightarrow \infty} \left( \Phi(k-x;x) - \Phi(-k-x;x) \right)
        \lim_{k\rightarrow \infty}\ln \left( \Phi(k-x;x) - \Phi(-k-x;x) \right)\\
    =& 1\times 0 = 0
\end{align*}
\end{proof}

\paragraph{(d)}
For $\theta_0 \in [-k, k]$ The reference prior $p_0(\theta)$ is given by
\begin{align*}
    p_0(\theta)
    &= \lim_{k\rightarrow \infty} \frac{\pi_\mathrm{Jeff}^k (\theta)}{\pi_\mathrm{Jeff}^k (\theta_0)}\\
    &= \lim_{k\rightarrow \infty} \frac{\frac{1}{2k}}{\frac{1}{2k}}\\
    &= 1
\end{align*}

Here the reference prior is independent from the choice of $\theta_0$ and is identical to the Jeffreys prior. This can be explained by the fact that the statistical model $\mathcal{P} \sim \mathcal{N}(\theta, 1)$ is under regularity conditions.
